%        File: desc.tex
%     Created: Tue Jun 12 08:00 PM 2018 C
% Last Change: Tue Jun 12 08:00 PM 2018 C
%
\documentclass[a4paper]{article}
\usepackage{amsfonts}
\usepackage{amsmath}
\usepackage{bbm}

\title{A Prototype Spiking Neural Network}
\author{Nathan Wycoff}

\begin{document}
\maketitle

\section{Existing Models}

\subsection{The Leaky Integrate and Fire Neuron}

The leaky linear integrate and fire neuron (LIF) is a first order system of ODE's:

$$
V_i' = -aV_i + I_i(t); i \in \{1, \ldots, h\}
$$

There is also a ``reset condition": if $V_i (\hat{t})= v_t$, with $v_t$ a known threshold value, then $\lim_{\epsilon\to 0^+} V(\hat{t} + \epsilon) = v_r$, with $v_r$ a reset value, generally just 0.

Let the set $F_i$ represent $\{t : V_i(t) = v_t\}$.

Then we can write the input current as such:

$$
I_i(t)= \sum_{j=1}^h w_{i,j} \sum_{\hat{t} \in F_j} \alpha(t - \hat{t})
$$

with $\alpha$ a kernel function which decays rather quickly. For our purposes, let's start with $\alpha(\Delta t) = a (\exp{\frac{-\Delta t}{b}} - \exp{\frac{-\Delta t}{c}}) \Theta(\Delta t)$ with $\Theta$ representing the heaviside function, constants $a, b, c$ known.

See ``./python/lif\_neuron.py" for an implementation using forward Euler.


\section{Model Description}

Each neuron's potential is described as a second order ODE:

\begin{equation} \label{eq1}
\begin{split}
V''(t) & = -aV(t) + \\
 & b_1 \mathbbm{1}_{V(t) > \alpha_1}\mathbbm{1}_{V(t) < \alpha_2}\mathbbm{1}_{V'(t) > 0} + \\
& -b_2 \mathbbm{1}_{V(t) > \alpha_2} + \\
& -b_3 \mathbbm{1}_{V(t) > \alpha_3} \mathbbm{1}_{V'(t) < 0} \mathbbm{1}_{V(t) < \alpha_2} + \\
& -b_4 \mathbbm{1}_{V(t) > \alpha_1} \mathbbm{1}_{V'(t) < 0} \mathbbm{1}_{V(t) < \alpha_3} + \\
& -b_5 \mathbbm{1}_{V(t) < \alpha_4} 
\end{split}
\end{equation}

Each line describes a separate component of the model. 

The first line tells us that the potential decays if there is no input from presynaptic neurons.

The second line causes acceleration if the potential gets too close to zero. The idea is that we don't want to potential to every cross below 0.

The third line causes acceleration when the potential crosses the threshold ($\alpha_1$).

The fourth line decelerates when the potential reaches the peak ($\alpha_2$).

The fifth line continues the deceleration throughout the spiking region (between $\alpha_1$ and $\alpha_2$).

\section{Choosing Model Parameters}

There are many parameters associated with this model that need to be decided upon.

Let's begin with the $b_i$'s. It is first desireable that the velocity of the potential is approximately the same after the spike as it was before. This may be achieved by setting $b_1 \Delta_1 + b_3 \Delta_3 - b_4 \Delta_4 = 0$ where $\Delta_i$ is the corresponding distance that the force acts upon.


\end{document}
