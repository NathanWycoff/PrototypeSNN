%        File: desc.tex
%     Created: Tue Jun 12 08:00 PM 2018 C
% Last Change: Tue Jun 12 08:00 PM 2018 C
%
\documentclass[a4paper]{article}
\usepackage{amsfonts}
\usepackage{amsmath}
\usepackage{bbm}

\title{A Prototype Spiking Neural Network}
\author{Nathan Wycoff}

\begin{document}
\maketitle

\section{Model Description}

Each neuron's potential is described as a second order ODE:

\begin{equation} \label{eq1}
\begin{split}
V''(t) & = -aV(t) + \\
& |V'(t)| \exp(1/V(t))  + \\
 & b \mathbbm{1}_{V(t) > \alpha_1}\mathbbm{1}_{V(t) < \alpha_2}\mathbbm{1}_{V'(t) > 0} + \\
& -b \mathbbm{1}_{V(t) > \alpha_2} + \\
& -b \mathbbm{1}_{V(t) > \alpha_1} + \mathbbm{1}_{V'(t) < 0}
\end{split}
\end{equation}

Each line describes a separate component of the model. 

The first line tells us that the potential decays if there is no input from presynaptic neurons.

The second line causes acceleration if the potential gets too close to zero. The idea is that we don't want to potential to every cross below 0.

The third line causes acceleration when the potential crosses the threshold ($\alpha_1$).

The fourth line decelerates when the potential reaches the peak ($\alpha_2$).

The fifth line continues the deceleration throughout the spiking region (between $\alpha_1$ and $\alpha_2$).


\end{document}
